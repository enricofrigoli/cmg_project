\section*{Introduction}

Large-scale metagenome assembly has the purpose of reconstructing bacterial genomes from shotgun sequencing experiments. In order to recompose entire genomes from reads that are only hundreds of nucleotides long, the sequences undergo a pipeline of assembling and quality control procedures that lead to the metagenome-assembled genomes (MAGs). MAGs are high quality single-genome bins, which are groups of contigs binned together because they seem to belong to the same genome. Bins are subject to strict quality controls to be considered genomes, and they should have high completeness (above 90\%) and low redundancy (below 5\%).\\

This project has as starting point a set of 31 high-quality single-genome bins: 29 of them are MAGs and two of them are reference genomes. The MAGs were obtained from five different published works on the gut microbiome \cite{origin1,origin2,origin3,origin4,origin5} and all of them were previously reunited in one species-level genome bin. After having checked that all the genomes belong to the same species, the focus of the project is aimed at extracting as much knowledge as possible from these datasets with the tools indicated in the \texttt{Methods} section. The genomes were firstly annotated to retrieve information about the proteins that can be translated from them, as well as the non-coding regions. Subsequently, a pangenome analysis was performed to identify the number of core genes (genes present in all strains) and to determine whether the bacterial species in analysis has an open or closed pangenome. The pangenome is the entire set of genes presented by all strains composing a species, and it can be considered open if its size increases indefinitely when new individual are added to the genomes bin, while it is closed if the number of genes composing it reaches a plateau when a great deal of genomes is analyzed. Thanks to this analysis, genes were classified in core, soft core, shell and cloud based on the percentage of prevalence in the set of MAGs. Moreover, a phylogenetic structure was determined based on the presence or absence of genes in each genome. A second type of phylogenetic analysis was performed thanks to the global alignment of the core genes, in order to produce a more accurate phylogenetic tree. In the end, the trees were compared with the host data associated with each MAG with the purpose of identifying some potential clusters. 