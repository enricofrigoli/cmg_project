\section*{Methods}

% roary commands: roary GFF/*.gff -f roary_out –e -p 8 -i 95 -cd 90
% we used PRANK as engine not MAFFT

%%% What software you used for each purpose, what parameters

A total of 31 HQ genomes (29 SGBs and 2 reference genome) were used in this study. 

\nocite{Tange2011a}


\subparagraph*{Pangenome Analysis}

%%%% PANGENOME ANALYSIS 
The species' pangenome was obtained using Roary v.3.7.0 (citation) taking Prokka annotation files (.gff) as input. 
The following parameters were given: \textbf{-i 95} to set the minimum percentage of identity
in the blastp alignement to 95\%, \textbf{-cd 90} to set the prevalence in \% MAGs for a gene
to be considered core, \textbf{-e} to produce the core gene alignement file using PRANK, \textbf{-p 8} to
specify the number of threads. 

The number of genes and related classification were retrieved from the output file
\textit{summary\_statistics.txt} and \textit{gene\_presence\_absence.csv}; plots were generated using two
plotting script (\href{https://github.com/sanger-pathogens/Roary/blob/master/bin/create_pan_genome_plots.R}{here}
and \href{https://github.com/sanger-pathogens/Roary/blob/master/contrib/roary_plots/roary_plots.py}{here}).


\subparagraph*{Phylogenetic Structure}

The resulting core gene alignement file (.aln) obtained with Roary was processed with FastTree v.2.1.10
using \textbf{-nt} as parameter (nucleotide alignement). ITOL v.6 (\href{https://itol.embl.de/}{Interactive Tree of Life})
was used to visualize both the tree obtained from Roary (.newick) and the one obtained from the core
gene alignement file processed with FastTree (.tre).




