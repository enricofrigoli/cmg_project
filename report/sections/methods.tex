\section*{Methods}

% roary commands: roary GFF/*.gff -f roary_out –e -p 8 -i 95 -cd 90
% we used PRANK as engine not MAFFT

%%% What software you used for each purpose, what parameters


\nocite{Tange2011a}

\subparagraph*{Taxonomic assignment}

The taxonomic characterization was performed with PhyloPhlAn v. 3.0.60
(\cite{phylophlan}). In particular, the selected tool was PhyloPhlAn metagenomic:
a tool that can assign genomes and MAGs to species-level genome bins by
computing their distance with a database of reference genomes. The parameters
chosen for the analysis were: \texttt{-nproc 4} to set the number of cores to be
used by the program, \texttt{-n 1} to set the number of best taxonomic
assignments to show, \texttt{--database\_update} to ensure the tool uses the
latest database version and \texttt{-d cmg2122} to set the database of markers
to be used. This analysis was performed on all the sequences of the set and the
tab separated file obtained as output contained the complete taxonomy, from 
kingdom to species, of each MAG.

\subparagraph*{Genome annotation}

The genomes were annotated with Prokka software v. 1.13 (\cite{prokka}).
The option \texttt{--kingdom Bacteria} was selected to specify that the genome
used ad input is bacterial, since Prokka also annotates archaeal and viral
genomes. This procedure was applied to all the sequences of the set. As output,
a series of files were obtained for each MAG and three of them were further
analyzed in this project: the annotation file (.gff) was used for the pangenome
analysis, the text and tab separated files were used to visualize some
statistics with R (\href{https://github.com/enricofrigoli/cmg_project/blob/main/Rscript/CMG_Rscript.md}{link to the script}).



\subparagraph*{Pangenome Analysis}

%%%% PANGENOME ANALYSIS 
The species' pangenome was obtained using Roary v.3.7.0 (\cite{Roary}) taking Prokka annotation files (.gff) as input. 
The following parameters were given: \texttt{-i 95} to set the minimum percentage of identity
in the blastp alignement to 95\%, \texttt{-cd 90} to set the prevalence in \% MAGs for a gene
to be considered core, \texttt{-e} to produce the core gene alignement file using PRANK (\cite{prank}), \texttt{-p 8} to
specify the number of threads. A core gene alignement using MAFFT ( \texttt{-n} parameter) was ran to check
the difference between the two alignement engines.

The number of genes and related classification were retrieved from the output file
\texttt{summary\_statistics.txt} and \texttt{gene\_presence\_absence.csv}; plots were generated using two
plotting script (\href{https://github.com/sanger-pathogens/Roary/blob/master/bin/create_pan_genome_plots.R}{link}
and \href{https://github.com/sanger-pathogens/Roary/blob/master/contrib/roary_plots/roary_plots.py}{link}).


\subparagraph*{Phylogenetic Structure}

The resulting core gene alignement file (.aln) obtained with Roary was manually checked with
Jalview v.2.11.21 and then processed with FastTree v.2.1.10 (\cite{fasttree})
using \texttt{-nt} as parameter (nucleotide alignement), which infers approximately-maximum-likelihood phylogenetic trees using the Jukes-Cantor model for nucleotide evolution.
The R package \texttt{ggtree} (\cite{ggtree1, ggtree2, ggtree3}) was used to plot
the tree with the selected metadata.





