\subsection*{Genome Annotation}
\graphicspath{{images/genomeAnnotation}}

% * Genome annotation: what functions are encoded in your MAGs?
% Hypothetical/annotated proteins

Genome annotation was performed with \textit{Prokka}. Thanks to this tool the number of coding sequences (CDS), and non-coding sequences found within each genome were retrieved. CDS are divided in hypothetical and known proteins, while non-coding sequences are comprehensive of rRNAs, tRNAs, tmRNAs (bifunctional transfer-messenger RNAs) and repeat regions. Moreover, \textit{Prokka} provides the symbol for each known protein and its sequence length.

The number of CDS in the set of MAGs goes to a minimum of 1988 to a maximum of 2544. For each genome, about half of the CDSs are non-characterized hypothetical proteins(figure). For what concerns non-coding sequences, the vast majority of them is represented by tRNAs (figure).
