\subsection*{Phylogenetic Structure and association with host data}
\graphicspath{{images/phylogeneticStructureHostData/}}


% Phylogenetic analysis and association with host data: comparison of phylogenetic
% trees based on accessory gene presence/absence or on core gene alignment.
% Do you detect clusters of strains? How do they associate with the metadata?


Significant differences in the topology were found between the tree obtained from the presence/absence of accessory genes
and the one built from the core gene alignement (figure \ref{fig:phylogenetic trees}). Instead,
MAFFT and PRANK produced core gene alignements that resulted in identical trees
(figure \ref{core alignement mafft tree}). 

Three main clusters of strains were seen in the tree obtained from the alignement, but the low number of samples and the high fragmentation of
information seen in the metadata impede the observation of finer clustering. 
For instance, no clustering with respect to the country nor to the disease were visible.


\begin{figure}[h!]
    \centering
    \begin{subfigure}[b]{0.7\textwidth}
        \centering
        \includegraphics[width=\textwidth]{tree_prank_age_BMI}
        \caption{}
        \label{fig:core alignment prank tree}
    \end{subfigure}
    \begin{subfigure}[b]{0.7\textwidth}
        \centering
        \includegraphics[width=\textwidth]{enriched_tree_Newick_final.png}
        \caption{}
        \label{fig:presence absence tree}
    \end{subfigure}
       \caption{}
       \label{fig:phylogenetic trees}
\end{figure}






