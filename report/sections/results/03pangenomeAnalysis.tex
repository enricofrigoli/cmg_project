\subsection*{Pangenome Analysis}
\graphicspath{{images/pangenomeAnalysis}}

% plots to be included:
% - conserved vs total genes




% Pangenome analysis:
% what’s the size of your pangenome? Is it closed or open?
% How many core and accessory genes?

Pangenome analysis found 4670 total genes (figure \ref{fig:pangenome pie}), 
among which 1017 were attributed to the \emph{core} (above 90\% prevalence in MAGs),
2074 to the \emph{shell} (from 15\% to 89\% prevalence), and 1579 were classified as \emph{cloud}
(from 0\% to 15\% prevalence). The results were robust with respect to many rounds of computation.

The number of conserved genes appears to reach a plateau (figure \ref{fig:conserved vs total})
when the number of MAGs increases, suggesting that this species has a closed pangenome.
This is further confermed by the trend of unique genes plotted against the number of
genomes (figure \ref{fig:unique vs new}).

The plot of pangenome frequencies (figure \ref{fig:pangeome frequency}) shows the typical shape observed in microbiome samples.
It is U shaped since there are conserved functions (core genes) that are present in every strain, and
other genes that are very specific and are present in only one or few strains (unique genes), with very few genes in between.

\begin{figure}[h!]	% to be fixed
     \centering
     \begin{subfigure}[b]{0.4\textwidth}
         \centering
         \includegraphics[width=\textwidth]{conserved_vs_total_genes}
         \caption{}
         \label{fig:conserved vs total}
     \end{subfigure}
     \hfill
     \begin{subfigure}[b]{0.4\textwidth}
         \centering
         \includegraphics[width=\textwidth]{unique_vs_new_genes}
         \caption{}
         \label{fig:unique vs new}
     \end{subfigure}
     \hfill
     \begin{subfigure}[b]{0.45\textwidth}
         \centering
         \includegraphics[width=\textwidth]{pangenome_pie}
         \caption{}
         \label{fig:pangenome pie}
     \end{subfigure}
     \hfill
     \begin{subfigure}[b]{0.45\textwidth}
         \centering
         \includegraphics[width=\textwidth]{pangenome_frequency}
         \caption{}
         \label{fig:pangeome frequency}
     \end{subfigure}
     \hfill
     \begin{subfigure}[b]{0.85\textwidth}
         \centering
         \includegraphics[width=\textwidth]{pangenome_matrix}
         \caption{}
         \label{fig:pangenome matrix}
     \end{subfigure}
        \caption{Three simple graphs}
        \label{fig:pangenome}
\end{figure}



% (https://github.com/sanger-pathogens/Roary/blob/master/bin/create_pan_genome_plots.R)
% 
% - Pangenome frequency plot
% - Presence and absence matrix and tree
% - Pangenome pie-chart (core, soft core, shell and cloud genes)
% (https://github.com/sanger-pathogens/Roary/blob/master/contrib/roary_plots/roary_plots.py)





















